\documentclass[12pts]{article}
\title{\textbf{NATIONAL INSTITUTE OF TECHNOLOGY, RAIPUR(C.G)}}
\usepackage{graphicx}
\graphicspath{{Images/}}


\author{SACHIN KUMAR\\sachin.1107sk@gmail.com\\Roll No: 21111047}
\date{February 27, 2022}
\begin{document}
\maketitle

\begin{figure}[h]
\centering
\includegraphics[scale=0.8]{nit.jpg}
\caption{National Institute of Technology, Raipur}
\end{figure}

\textbf{ASSIGNMENT-6 OF BASIC BIO-MEDICAL ENGINEERING}\\
\centering
UNDER THE SUPERVISION OF DR. SAURABH GUPTA SIR\\

\clearpage



\section*{5 Solutions to Covid-19 provided by Biomedical Engineers}



This section provides an overview of how biomedical engineering is contributing to the management of the COVID-19 pandemic.
Here we are discuusing about the solutions how covid-19 coulb be in control and what Bio-Medical Engineers have done during this unexpected pandemic crisis.
The use of medical devices in the COVID pandemic is the unfortunate indication that the patients are displaying severe respiratory distress symptoms and need a form of assistance to breathe.
In india we all have faced the problem whic has been occurred by the lack of oxygen cylinders.\\
The role of a Biomedical Engineer includes designing biomedical equipment and devices to aid the recovery or improve the health of individuals. This can include internal devices, such as stents or artificial organs, or external devices, such as braces and supports (orthotics). It can also include creating and adapting medical equipment. It’s a role that requires excellent knowledge of computing, biology and engineering, an inventive nature, and good problem solving skills.
According to a report published on January 24, 2020, patients infected with corona-
virus have several characteristics in common: fever, cough, and fatigue, while diarrhea and problebms of breath are found to be rare. To overcome on this crisis our biomedical engineers helps us to maintain sustainability in the nation as well as world. 
Coronaviruses are a type of virus that is known to affect the respiratory tract of animals as well as humans. They are part of the Coronavirinae subfamily which falls under the Coronaviridae family. Some of the most common types of Coronavirus include:\\

\begin{enumerate}
\item 229E (alpha coronavirus)
\item NL63 (alpha coronavirus
\item OC43 (beta coronavirus)
\item HKU1 (beta coronavirus)
\end{enumerate}
\begin{figure}[h]
\includegraphics[scale=0.17]{type.jpg}
\includegraphics[scale=0.5]{how.jpg}
\caption{Types of corona virus and How it affects you?}
\end{figure}

\begin{itemize}
\item \textbf{ Covid-19 Cases: 43.4Cr}
\item \textbf{ Covid-19 Cases in India: 4.29Cr}
\item \textbf{Total deaths today onwards: 59.4L}
\item \textbf{Total deaths today onwards in India: 5.14L}
\end{itemize} 


\subsection*{Solution-1: Keeping oxygen level proper}

The first form for mild respiratory insufficiency is usually the supply of extra oxygen through a nasal cannula or a more intrusive face mask. Most of the time, the oxygen comes in the form of cylinders, either small for portability or large for stationary patients and longer-term supply.

Oxygen concentrators represent an attractive alternative to tanks although this is not typically in use while caring for COVID-19 patients in hospital settings. Oxygen concentrators extract oxygen from the air on demand and supply it directly to the patient. Concentrators come in a variety of sizes from a portable shoulder bag form factor, to higher capacity stationary machines for patients who need oxygen 24/7(means everytime monitoring whenever they need).

Variants of oxygen supply include high flow nasal oxygen (HFNO) which delivers warmed and humidified oxygen, to avoid the drying of airways, at high flow rates - typically tens of litres/min) at body temperature and up to 100 percent Relative Humidity(RH) and 100 percent oxygen.

\subsection*{Solution-2:  On keeping patients on Ventilators}
\begin{figure}[h]
\centering
\includegraphics[scale=0.4]{venti.jpg}
\caption{Key features of Ventilator}
\end{figure}

Patients who cannot breathe spontaneously need to be put on a ventilator. Ventilators are capable of replacing the breath function and patients in an advanced state of respiratory. If we will be providing the enough amount diagnosis to the patients thus they will be gradually getting better.

Ventilators are capable of replacing the breath function and patients in an advanced state of respiratory distress are usually intubated and sedated at the beginning of the treatment. They are complex systems providing the healthcare professionals with a lot of flexibility to adapt the assisted breathing settings and to be able to wean recovering patients off the ventilator gradually.

Modern ventilators are typically closed loop pressure controlled and capable of detecting spontaneous breathing to synchronise assistance for recovering patients. They also enable the control of the composition of the gas the patient breathes from normal air to 100 percent oxygen, usually taking their supply from the hospital’s gas supply network but can also be coupled to oxygen tanks or oxygen concentrators if used in a setting where there is no gas network.



\subsection*{Solution-3: Time to Time Patient monitoring}
As we all know that how we are having the lack of doctors and good doctors thus everytime a doctor couldn't be able to monitor the patients health everytime thus Bio-Meedical Engineers designed many type of devices like Ventilator, infusion pump, Blood warmer,syringe pump etc.\\
An essential element of the ICU equipment is the monitoring equipment that keeps track of some of the patient vitals especially when they are ventilated and sedated but also during their recovery phase to ensure the regime of ventilation is optimised for their condition. Ventilators already provide their set of patient parameters, but usually patient monitors are separate devices as they continue to be useful after the patient can resume breathing on their own unassisted.

One of the key parameters for COVID-19 patient is the amount of oxygen in their bloodstream (SpO2), measured by pulse oximetry which uses optics within a finger clamp. Pulse oximetry tends to be used for the duration of the patient’s stay in ICU.

Modern patient monitors provide many more patient parameters all the way to breathing waveforms to enable clinicians to fine tune their care of the patients.

\subsection*{Solution-4: To increase awareness in every type of areas }

Here we are talking about how people should follow the instructions which are given by the WHO or Doctors or Bio-Medical Engineers.
\begin{itemize}
\item Washing your hands frequently (for at least 20 seconds) with soap and water or a hand
sanitizer that contains at least 60% alcohol.

\item  Avoiding touching your face (particularly your eyes, nose, and mouth).
\item Staying home as much as possible, even if you don’t feel sick.
\item Avoiding crowds and gatherings of 10 or more people.
\item Keeping 6 feet of distance between yourself and others when out.
\item Getting plenty of sleep, which helps support your immune system.
\end{itemize}


\subsection*{Solution-5: How to Prevent Coronavirus infection transmission?}

\begin{figure}[h]
\centering
\includegraphics[scale=0.35]{infe.jpg}
\caption{Peoples wearing masks during covid crisis}
\end{figure}

In case you are diagnosed with Coronavirus, your healthcare provider will decide whether you need to be hospitalised or can remain at home. In case you’re asked to stay at home, you’ll be monitored by the state or central healthcare department. More commonly state healthcare takes part in these type of pandemics.
It can transmit in peoples by the primary mode of COVID-19 transmission is through respiratory droplets generated when an infected person coughs, sneezes, or talks. Droplets that accumulates on the eyes, nose, or mouth of a person in close proximity{coming closer} leads to the transmission of infection. Transmission can also occur by touching the face with contaminated hands. Respiratory droplets do not remain suspended in the air for long; hence, a distance of six feet away from an infected person may be considered safe.\\
$Coronaviruses$ may contaminate metal, glass, or plastic surfaces that may remain infective for several days. Contact with such contaminated surfaces (fomites) and subsequent transfer to the face by touch may also be an important mode of transmission.

\begin{itemize}
\item Wear a mask – Whenever you’re outside in a public place, you should always wear a facemask.
\item Cover your mouth while coughing and sneezing – You should     also cover your mouth with a tissue while coughing or sneezing.
\item Avoid sharing personal items.
\item Clean your hands time to time.
\end{itemize}

\end{document}